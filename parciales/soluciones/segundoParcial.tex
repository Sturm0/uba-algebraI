\documentclass{article}
\usepackage{ifthen}
\usepackage{amssymb}
\usepackage{multicol}
\usepackage{graphicx}
\usepackage[absolute]{textpos}
\usepackage{amsmath, amscd, amssymb, amsthm, latexsym}
\usepackage{xspace,rotating,dsfont,ifthen}
\usepackage[spanish,activeacute]{babel}
\usepackage[utf8]{inputenc}
\usepackage{pgfpages}
\usepackage{pgf,pgfarrows,pgfnodes,pgfautomata,pgfheaps,xspace,dsfont}
\usepackage{listings}
\usepackage{multicol}
\usepackage{todonotes}
\usepackage{url}
\usepackage{float}
\usepackage{framed,mdframed}
\usepackage{cancel}

\usepackage[strict]{changepage}


\makeatletter


\newcommand\hfrac[2]{\genfrac{}{}{0pt}{}{#1}{#2}} %\hfrac{}{} es un \frac sin la linea del medio

\newcommand\Wider[2][3em]{% \Wider[3em]{} reduce los m\'argenes
\makebox[\linewidth][c]{%
  \begin{minipage}{\dimexpr\textwidth+#1\relax}
  \raggedright#2
  \end{minipage}%
  }%
}


\@ifclassloaded{beamer}{%
  \newcommand{\tocarEspacios}{%
    \addtolength{\leftskip}{4em}%
    \addtolength{\parindent}{-3em}%
  }%
}
{%
  \usepackage[top=1cm,bottom=2cm,left=1cm,right=1cm]{geometry}%
  \usepackage{color}%
  \newcommand{\tocarEspacios}{%
    \addtolength{\leftskip}{3em}%
    \setlength{\parindent}{0em}%
  }%
}

\usepackage{caratula}
\usepackage{enumerate}
\usepackage{hyperref}
\usepackage{graphicx}
\usepackage{amsfonts}
\usepackage{enumitem}
\usepackage{amsmath}

\decimalpoint
\hypersetup{colorlinks=true, linkcolor=black, urlcolor=blue}
\setlength{\parindent}{0em}
\setlength{\parskip}{0.5em}
\setcounter{tocdepth}{3} % profundidad de indice
\setcounter{section}{1} % nro de section
\renewcommand{\thesubsubsection}{\thesubsection.\Alph{subsubsection}}
\graphicspath{ {images/} }

% End latex config

\begin{document}

\titulo{Segundo parcial 30/11/2021}
\fecha{2do cuatrimestre 2021}
\materia{Álgebra I}
\integrante{Yago Pajariño}{546/21}{ypajarino@dc.uba.ar}

%Carátula
\maketitle
\newpage

%Indice
\tableofcontents
\newpage

% Aca empieza lo propio del documento
\section{Segundo parcial Álgebra I}
\subsection{Ejercicio 1}

Busco $ (a,b) \in \mathbb{Z}^2: 51a+33b = 21 \wedge 8a \equiv b(49) $

Primero busco soluciones para la ecuación diofántica $ 51a+33b = 21 $

\textbf{1) verificar que existe solución}

Existe solución $ (a,b) \in \mathbb{Z}^2 \iff (51:33) | 21 $

Luego,
\begin{align*}
    (51:33) &= (3.17: 3.11) \\
    &= 3 \\
\end{align*}
Como $ 3|21 $ existe solución a la ecuación.

\textbf{2) coprimizar}
\begin{align*}
    51a+33b = 21 &\iff 3.17.a + 3.11.b = 3.7 \\
    &\iff 17a + 11b = 7 \\
\end{align*}

\textbf{3) busco solución particular}

Por propiedades del MCD, se que existen $ (s,t) \in \mathbb{Z}^2: (17:11) = s.17 + t.11 $

Dado que 17 y 11 son ambos primos, en particular $ (17:11) = 1 \implies \exists (s,t) \in \mathbb{Z}^2: 1 = s.17 + t.11 $

A ojo veo que $ 2.17 + (-3).11 = 34 - 33 = 1 $

Por lo tanto, $ 1 = 2.17 + (-3).11 \iff 7 = 14.17 + (-21).11 $

Así encuentro que $ S_p = (14,-21) $ es solución particular de la ecuación.

\textbf{4) busco solución del homogeneo asociado}
\begin{align*}
    17a + 11b = 0 \iff a = 11k \wedge b = -17k; \forall k \in \mathbb{Z}
\end{align*}
Luego $ S_0 = (11k, -17k) $ es solución al homogeneo asociado.

\textbf{5) busco todas las soluciones}

Con lo hallado obtengo que,

\begin{align*}
    S &= S_0 + S_p \\ 
    &= (11k; -17k) + (14; -21) \\ 
    &= (11k+14; -17k-21); k \in \mathbb{Z}
\end{align*}
Y así, $S$ es el conjunto de soluciones a la ecuación diofántica.

\textbf{6) verifico el conjunto solución}
\begin{align*}
    51a+33b = 21 &\iff 51(11k+14) + 33(-17k-21) = 21 \\ 
    &\iff 561k + 714 - 561k - 693 = 21 \\
    &\iff 714 - 693 = 21 \\
    &\iff 21 = 21 \\
\end{align*}
Como se quería verificar.

Ahora utilizo la otra restricción. Sabiendo que $ (a,b) = (11k+14; -17k-21) $ son soluciones de la ecuación diofántica, busco los $(a,b) $ tales que $ 8a \equiv b(49) $
\begin{align*}
    8a \equiv b(49) &\iff 8(11k+14) \equiv -17k-21(49) \\
    &\iff 88k + 112 \equiv -17k-21(49) \\
    &\iff 88k + 17k \equiv -21-112(49) \\
    &\iff 105k \equiv -133(49) \\
    &\iff 7k \equiv 14(49) \\
    &\iff k \equiv 2(7) \\
\end{align*}
Entonces para que se cumpla la segunda restricción, necesito que $ k = 7h + 2; h \in \mathbb{Z} $ 

Por lo tanto si $ k = 7h + 2 $ ,
\begin{align*}
    (a,b) &= (11k+14; -17k-21) \\
    &= (11(7h + 2)+14; -17(7h + 2)-21) \\
    &= (77h + 36; -119h - 55) \\
\end{align*}
Rta.: $ \{ (a,b) \in \mathbb{Z}^2 / a = 77h+36 \wedge b = -119h-55 \wedge h \in \mathbb{Z} \} $

\subsection{Ejercicio 2}

Busco el resto de dividir a $ 8^{3^n-2} $ por $20$

Usando congruencia, $ 8^{3^n-2} \equiv a (20) $

Por el teorema chino del resto, se que existe una única solución mod 20 que satisface $ \begin{cases}
    8^{3^n-2} = x (4) \\
    8^{3^n-2} = y (5) \\
\end{cases} $ pues $(4:5) = 1$

\textbf{Busco x}

Se que $ 8 \equiv 0(4) $ pero,
\begin{align*}
    8^{3^n-2} \equiv 0 (4) \iff 3^n-2 \geq 1
\end{align*}
Y $ 3^n-2 \geq 1; \forall n \in \mathbb{N} $

Luego $ x = 0 $

\textbf{Busco y}

Por el Pequeño Teorema de Fermat, dados $ a \in \mathbb{Z}; p \text{ primo }; a \perp p \implies a^{p-1} \equiv 1(p) $

En particular, $ 8\perp 5 \wedge 5 $ primo $ \implies 8^4 \equiv 1(5) $

Usando el algoritmo de división se que, $ 3^n-2 = 4j + r_4(3^n-2); j \in \mathbb{Z} $

Por lo tanto,
\begin{align*}
    8^{3^n-2} &= 8^{4j + r_4(3^n-2)} \\
    &= (8^4)^j \cdot 8^{r_4(3^n-2)} \\
    &\equiv 8^{r_4(3^n-2)} (5) \\
\end{align*}
Luego, $ r_4(3^n-2) \implies 3^n-2 \equiv (-1)^n + 2 (4) $

\begin{itemize}
    \item n par $ \implies r_4(3^n-2) = 3 $
    \item n impar $ \implies r_4(3^n-2) = 1 $
\end{itemize}
Y por lo tanto
\begin{itemize}
    \item n par $ \implies 8^{r_4(3^n-2)} \equiv 8^3 \equiv 3^3 \equiv 2 (5) $
    \item n impar $ \implies 8^{r_4(3^n-2)} \equiv 8^1 \equiv 3 (5) $
\end{itemize}

Así, $ y = \begin{cases}
    2 & n \equiv 0 (2) \\
    3 & n \equiv 1 (2) \\
\end{cases} $

Volviendo al sistema de ecuaciones con $x$ e $y$ hallados me quedan dos sistemas,:

$ S_1 = \begin{cases}
    8^{3^n-2} \equiv 0 (4) \\
    8^{3^n-2} \equiv 2 (5)
\end{cases} n \equiv 0(2)$

$ S_2 = \begin{cases}
    8^{3^n-2} \equiv 0 (4) \\
    8^{3^n-2} \equiv 3 (5)
\end{cases} n \equiv 1(2)$

Por TCR ya enunciado existe una única solución de $S_1$. A ojo veo que $ 8^{3^n-2} \equiv 12(20) $ es solución de $S_1$

Por TCR ya enunciado existe una única solución de $S_2$. A ojo veo que $ 8^{3^n-2} \equiv 8(20) $ es solución de $S_2$

Rta.: $ r_{20}(8^{3^n-2}) = \begin{cases}
    12 & n \equiv 0(2) \\
    8 & n \equiv 1(2)
\end{cases} $

\subsection{Ejercicio 3}

\subsubsection*{Pregunta i}

Por propiedades de las raíces multiples: $ a \in \mathbb{Q} \text{ es raíz multiple de f } \iff f(a) = 0 \wedge f'(a) = 0 $

Luego,
\begin{align*}
    f'(a) = 0 &\iff 6x^5 - 5(a-1)a^4 - 4(a-1)a^3 - 3(a-1)a^2 - 2(a+2)a + 2a - 2 = 0 \\
    &\iff 6x^5 - 5a^5 + 5a^4 - 4a^4 + 4a^3 - 3a^2 - 2a^2 - 4a + 2a - 2 = 0 \\
    &\iff a^5 + a^4 + a^3 + a^2 - 2a -2 = 0 \\
\end{align*}
Busco los $ a \in \mathbb{Q}: a^5 + a^4 + a^3 + a^2 - 2a -2 = 0 $

Luego por el lema de Gauss se que: sea $ p \in \mathbb{Z}[x] $ un polinomio, $ f(\frac{c}{d}) = 0 \implies c | a_0 \wedge d | cp(f) $

Por lo tanto $ a \in \{ \pm 1, \pm 2 \} $

Evalúo $ f' $ en los posibles candidatos:
\begin{itemize}
    \item $ f'(-1) = -1+1-1+1+2-2 = 0 $ 
    \item $ f'(1) = 1+1+1+1-2-2 = 0 $ 
    \item $ f'(2) = 32+16+8+4-4-2 \neq 0 $ 
    \item $ f'(-2) = -32+16-8+4+4-2 \neq 0 $ 
\end{itemize}
Luego $ a = \pm 1 $ será raíz multiple de f $ \iff f(a) = 0 $

Evalúo $f$ en $ a = 1 $
\begin{align*}
    f(1) &= 1^6 - (1-1)1^5 - (1-1)1^4 - (1-1)1^3 - (1+2)1^2 + 2(1-1)1 + 2 \\
    &= 1-3+2 \\
    &= 0 \\
\end{align*}
Luego $ a = 1 $ es raíz multiple de f.

Evalúo $f$ en $ a = -1 $
\begin{align*}
    f(-1) &= (-1)^6 - (-1-1)(-1)^5 - (-1-1)(-1)^4 - (-1-1)(-1)^3 -(-1+2)(-1)^2 + 2(-1-1)(-1) + 2(-1) \\
    &= 1 - (-2)(-1) - (-2).1 - (-2)(-1) - 1 + 2(-2)(-1) - 2 \\
    &= 1-2+2-2-1+4-2 \\
    &= 0 \\
\end{align*}
Luego $ a = -1 $ es raíz multiple de f.

Rta.: $ a \in \{ -1, 1 \} $

\subsubsection{Pregunta ii}

Me piden factorizar $f$ con $ a = -1 $

Con $ a = -1 $ f queda:
\begin{align*}
    f &= x^6 - (-1-1)x^5 - (-1-1)x^4 - (-1-1)x^3 - (-1+2)x^2 + 2(-1-1)x + 2(-1) \\
    &= x^6 + 2x^5 + 2x^4 + 2x^3 - x^2 - 4x - 2 \\
\end{align*}
Por el inciso anterior se que $ a = -1 $ es raíz multiple de f.
Aca se puede hacer Ruffini dos veces que fue lo que hice en el parcial, o usar el algoritmo de división con $ (x+1)^2 $

Haciendo Ruffini o por algoritmo de división se obtiene: $ f = (x+1)^2(x^4+x^2-2) $

Defino $ g = x^4+x^2-2 $ y busco sus raíces.

Se ve a simple vista que $ g(\pm 1) = 0 $

Luego usando Ruffini se obtiene que $ g = (x-1)(x+1)(x^2 + 2) $

Defino $ h = x^2 + 2 $ y busco sus raíces.

$ h(a) = 0 \iff a^2 = -2 \iff a = \pm \sqrt[]{2}i $

Luego $ h = (x-\sqrt[]{2}i)(x+\sqrt[]{2}i) $

Con todo lo hallado, armo las factorizaciones.

$ f = (x+1)^3(x-1)(x-\sqrt[]{2}i)(x+\sqrt[]{2}i) $ es la factorización en irreducibles en $ \mathbb{C}[x] $ pues todos los factores son mónicos de grado 1. 

$ f = (x+1)^3(x-1)(x^2 + 2) $ es la factorización en irreducibles en $ \mathbb{Q}[x]; \mathbb{R}[x] $ pues $ (x+1)(x-1) $ con mónicos de grado 1 y $ (x^2+2) $ tiene grados dos y no tiene raíces en $ \mathbb{Q}; \mathbb{R} $

\subsection{Ejercicio 4}

Busco un polinomio $ f \in \mathbb{Q}[x] $ mónico de grado mínimo tal que:
\begin{enumerate}
    \item $ 1+\sqrt[]{2} $ sea raíz de $f$
    \item $ x^2(x+1) | (f:f') $
    \item $ f(1) = 20 $
\end{enumerate}
Por (1) se que $ 1+\sqrt[]{2} $ es raíz, pero por propiedades de $ f \in \mathbb{Q}[x] $ se que si $ a+b\sqrt[]{d} $ es raíz, $ a-b\sqrt[]{d} $ también lo es.

En particular, $ 1+\sqrt[]{2} $ es raíz de f $ \iff 1-\sqrt[]{2} $ es raíz de f.

Luego se que $ (x-(1+\sqrt[]{2}))(x-(1-\sqrt[]{2})) | f $

$ (x-(1+\sqrt[]{2}))(x-(1-\sqrt[]{2})) = x^2-2x-1 $  

Por (2) se que $ x^2 | f $ y $ x^2 | f' $, como busco $f$ de menor grado y $ mult(0,f) \geq 2 \implies mult(0,f) = 3 $

Por (2) se que $ (x+1) | f $ y $ (x+1) | f' $, como busco $f$ de menor grado y $ mult(-1,f) \geq 1 \implies mult(-1,f) = 2 $

Luego $ x^3(x+1)^2 | f $

Juntando lo hallado, $ f = x^3(x+1)^2(x^2 - 2x - 2) $ es el de menor grado que cumple (1) y (2)

Además, me piden que $ f(1) = 20 $ esto solo lo puedo lograr agregando un nuevo termino, podría lograrlo agregando una constante, pero el polinomio dejaría de ser mónico. Luego,
\begin{align*}
    f(1) = 20 &\iff 1^3(1+1)^2(1^2 - 2.1 - 2)(1-a) = 20 \\
    &\iff 4(-3)(1-a) = 20 \\
    &\iff 1-a = -\frac{20}{12} \\
    &\iff a = \frac{8}{3} \\
\end{align*}
Rta.: $ f = x^3(x+1)^2(x^2-2x-2)(x-\frac{8}{3}) $ es el polinomio mónico de menor grado que cumple lo pedido.

\end{document}
