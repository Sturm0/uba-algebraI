\documentclass{article}
\usepackage{ifthen}
\usepackage{amssymb}
\usepackage{multicol}
\usepackage{graphicx}
\usepackage[absolute]{textpos}
\usepackage{amsmath, amscd, amssymb, amsthm, latexsym}
\usepackage{xspace,rotating,dsfont,ifthen}
\usepackage[spanish,activeacute]{babel}
\usepackage[utf8]{inputenc}
\usepackage{pgfpages}
\usepackage{pgf,pgfarrows,pgfnodes,pgfautomata,pgfheaps,xspace,dsfont}
\usepackage{listings}
\usepackage{multicol}
\usepackage{todonotes}
\usepackage{url}
\usepackage{float}
\usepackage{framed,mdframed}
\usepackage{cancel}

\usepackage[strict]{changepage}


\makeatletter


\newcommand\hfrac[2]{\genfrac{}{}{0pt}{}{#1}{#2}} %\hfrac{}{} es un \frac sin la linea del medio

\newcommand\Wider[2][3em]{% \Wider[3em]{} reduce los m\'argenes
\makebox[\linewidth][c]{%
  \begin{minipage}{\dimexpr\textwidth+#1\relax}
  \raggedright#2
  \end{minipage}%
  }%
}


\@ifclassloaded{beamer}{%
  \newcommand{\tocarEspacios}{%
    \addtolength{\leftskip}{4em}%
    \addtolength{\parindent}{-3em}%
  }%
}
{%
  \usepackage[top=1cm,bottom=2cm,left=1cm,right=1cm]{geometry}%
  \usepackage{color}%
  \newcommand{\tocarEspacios}{%
    \addtolength{\leftskip}{3em}%
    \setlength{\parindent}{0em}%
  }%
}

\usepackage{caratula}
\usepackage{enumerate}
\usepackage{hyperref}
\usepackage{graphicx}
\usepackage{amsfonts}
\usepackage{enumitem}
\usepackage{amsmath}

\decimalpoint
\hypersetup{colorlinks=true, linkcolor=black, urlcolor=blue}
\setlength{\parindent}{0em}
\setlength{\parskip}{0.5em}
\setcounter{tocdepth}{2} % profundidad de indice
\setcounter{section}{4} % nro de section
\renewcommand{\thesubsubsection}{\thesubsection.\Alph{subsubsection}}
\graphicspath{ {images/} }

% End latex config

\begin{document}

\titulo{Práctica 5}
\fecha{2do cuatrimestre 2021}
\materia{Álgebra I}
\integrante{Yago Pajariño}{546/21}{ypajarino@dc.uba.ar}

%Carátula
\maketitle
\newpage

%Indice
\tableofcontents
\newpage

% Aca empieza lo propio del documento
\section{Práctica 5}

\subsection{Ejercicio 1}

\subsubsection{Pregunta i}

Busco los $ (a,b) \in \mathbb{Z}^2 $ tales que $ 7a+11b = 10 $

\textbf{Verifico que existe solución}

Dado que $ (7:11) = 1 \implies 1|10 \implies $ existe solución.

\textbf{Busco una solución particular}

Por propiedades del MCD se que existen $ (s,t) \in \mathbb{Z}^2 $ tales que:
\begin{align*}
    7.s + 11.t &= 1 \\
    7.(-3) + 11.2 &= 1 \\
    7.(-3).10 + 11.2.10 &= 1.10 \\
    7.(-30) + 11.20 &= 10 \\
    -210+220 &= 10 \\
\end{align*}

Luego $ (-30,20) $ es solución particular.

\textbf{Solución del homogeneo asociado}

$ 7.a+11.b = 0 \iff 7.a = -11.b \iff -11|7.a \iff -11|a \iff a = -11.k $

$ 7.a+11.b = 0 \iff 7.a = -11.b \iff 7(-11.k) = 11.b \iff b = 7.k$

Luego $ (-11.k, 7.k) $ es solución del homogeneo asociado, $ \forall k \in \mathbb{Z} $

\textbf{Solución general}

Uniendo las dos soluciones halladas previamente,

$ S = (-11.k, 7.k) + (-30,20) = (-11.k-30,7.k+20); \forall k \in \mathbb{Z} $

\textbf{Verifico}

Sea $ (a,b) = (-11.k-30,7.k+20) $ luego,
\begin{align*}
    7a+11b = 10 &\iff 7(-11.k-30) + 11(7.k+20) = 10\\ 
    &\iff 7(-11.k-30) + 11(7.k+20) = 10 \\
    &\iff 7.-11.k-210 + 11.7.k+220 = 10 \\
    &\iff -210 + 220 = 10 \\
\end{align*}

Verificado.

\subsubsection{Pregunta ii}

Busco los $ (a,b) \in \mathbb{Z}^2 $ tales que $ 20a+16b = 30 $

\textbf{Verifico que existe solución}

$ (20:16) = 4 \wedge 4\not | 30 $

Por lo tanto no hay solución en $ \mathbb{Z}^2 $ para la ecuación.

\subsubsection{Pregunta iii}

Busco los $ (a,b) \in \mathbb{Z}^2 $ tales que $ 39a-24b = 6 $

\textbf{Verifico que existe solución}

$ (39:24) = 3 \wedge 3|6 $ luego existe solución en $ \mathbb{z}^2 $

\textbf{Coprimizar}

Dado que el MCD es disinto a 1, debo coprimizar la ecuación para no perder soluciones.

$ 20a+16b = 30 \leftrightsquigarrow 13a-8b = 2$

\textbf{Busco una solución particular}

$ 13.(2) - 8.(3) = 26-24 = 2 $

Luego $ (2,3) $ es solución particular.

\textbf{Solución del homogeneo asociado}

$ 13a -8b = 0 \iff 13a = 8b \implies 13|8b \iff 13|b \iff b = 13k $

$ 13a -8b = 0 \iff 13a = 8b \iff 13a = 8(13k) \iff a = 8k $

Luego $ (8k,13k) $ es solución del homogeneo asociado, $ \forall k \in \mathbb{Z} $

\textbf{Solución general}

Uniendo las dos soluciones halladas previamente,

$ S = (8k,13k) + (2,3) = (8k+2, 13k+3); \forall k \in \mathbb{Z} $

\textbf{Verifico}

Sea $ (a,b) = (8k+2, 13k+3) $ luego,
\begin{align*}
    39a-24b = 6 &\iff 39(8k+2)-24(13k+3) = 6\\ 
    &\iff 39.8k+78-24.13k-72 = 6\\ 
    &\iff 78-72 = 6\\ 
\end{align*}

Verificado.

\subsubsection{Pregunta iv}

Busco los $ (a,b) \in \mathbb{Z}^2 $ tales que $ 1555a-300b = 11 $

\textbf{Verifico que existe solución}

$ (1555:300) = 5 \wedge 4\not | 5 $

Por lo tanto no hay solución en $ \mathbb{Z}^2 $ para la ecuación.

\subsection{Ejercicio 2}

Primero busco soluciones $ (a,b) \in \mathbb{Z}^2 $ para la ecuación $ 33a+9b=120 $

\textbf{Verifico que existe solución}

$ (33:9) = 3 \wedge 3|120 \implies $ existe solución.

\textbf{Coprimizar}

Dado que el MCD es disinto a 1, debo coprimizar la ecuación para no perder soluciones.

$ 33a+9b = 120 \leftrightsquigarrow 11a+3b=40$

\textbf{Busco una solución particular}

Por propiedades del MCD se que existen $ (s,t) \in \mathbb{Z}^2 $ tales que:
\begin{align*}
    11.s + 3.t &= 1 \\
    11.(2) + 3.(-7) &= 1 \\
    11.2.(40) + 3.(-7).(40) &= 1.(40) \\
    11.80 + 3.(-280) &= 40 \\
\end{align*}
Luego $ (80,-280) $ es solución particular.

\textbf{Solución del homogeneo asociado}

$ 11a +3b = 0 \iff 11a = -3b \implies 11|-3b \implies 11|b \implies b=11k $

$ 11a +3b = 0 \iff 11a = -3b \implies 11a = -3(11k) \implies a = -3k $

Luego $ (-3k,11k) $ es solución del homogeneo asociado, $ \forall k \in \mathbb{Z} $

\textbf{Solución general}

Uniendo las dos soluciones halladas previamente,

$ S = (-3k,11k) + (80,-280) = (-3k+80, 11k-280); \forall k \in \mathbb{Z} $

Luego tengo definidas las restricciones para a y b de tal forma que cumplan con la diofántica, ahora uso los datos de divisibilidad.

$ a = -3k+80 \implies -3k+80 \equiv 0(4) \implies k \equiv 0(4)$

$ b = 11k-280 \implies 11k-280 \equiv 0(8) \implies k \equiv 0(8)$

Pero $ k \equiv 0(8) \iff k \equiv 0(4) $

Luego $ k = 8n \implies a = 3(8n) + 80 \wedge b=11(8n)-280; n \in \mathbb{Z}$

Rta.: $ (a,b) = (24n+80, 88n-280); \forall n \in \mathbb{Z} $

\subsection{Ejercicio 3}

Busco los $ (a,b) \in \mathbb{Z}^2 $ que cumplen $ 39a+48b=135 $

\textbf{Verifico que existe solución}

$ (39:135) = 3 \wedge 3|135 \implies $ existe solución.

\textbf{Coprimizar}

Dado que el MCD es disinto a 1, debo coprimizar la ecuación para no perder soluciones.

$ 39a+48b=135 \leftrightsquigarrow 13a+16b=45$

\textbf{Busco una solución particular}

$ (225,-180) $ es solución particular.

\textbf{Solución del homogeneo asociado}

$ 13a +16b = 0 \iff 13a = -16b \implies 13|-16b \implies 13|b \implies b=13k $

$ 13a +16b = 0 \iff 13a = -16b \implies 13a = -16(13k) \implies a = -16k $

Luego $ (-16k,13k) $ es solución del homogeneo asociado, $ \forall k \in \mathbb{Z} $

\textbf{Solución general}

Uniendo las dos soluciones halladas previamente,

$ S = (-16k,13k) + (225,-180) = (-16k+225,13k-180); \forall k \in \mathbb{Z} $

Luego, \\
$ a \geq 0 \implies -16k+225 \geq 0 \implies 16k \leq 225 \implies k \leq \frac{225}{16} \implies k \leq 14 $ \\
$ b \geq 0 \implies 13k-180 \geq 0 \implies 13k \geq 180 \implies k \geq \frac{180}{13} \implies k \geq 14 $

Luego $ 14 \leq k \leq 14 \implies k = 14 \implies $ se compran 1 unidad de a y 2 de b, gastando 135 pesos.

\subsection{Ejercicio 4}

\begin{enumerate}
    \item $ 17x \equiv 3(11) \iff 6x \equiv 3(11) \iff 2.6 \equiv 2.3(11) \iff x \equiv 6(11) $
    \item $ 56x \equiv 28(35) \iff 21x \equiv 28(35) \iff 3x \equiv 4(5) \iff 6x \equiv 8(5) \iff x \equiv 3(5) $
    \item $ 56x \equiv 2(884) $ No tiene solución pues $ (56:884) = 4 \not | 2 $
    \item $ 78x \equiv 30(12126) \iff 13x\equiv 5(2021) \iff 311.13x \equiv 311.5(2021) \iff x \equiv 1551(2021) $
\end{enumerate}

\subsection{Ejercicio 5}

Primero resuelvo la diofántica: busco $ (a,b) \in \mathbb{Z}^2: 28a+10b = 26 $

\textbf{Verifico que existe solución}

$ (28:10) = 2 \wedge 2|26 \implies $ existe solución.

\textbf{Coprimizar}

Dado que el MCD es disinto a 1, debo coprimizar la ecuación para no perder soluciones.

$ 28a+10b =26 \leftrightsquigarrow 14a+5b=13$

\textbf{Busco una solución particular}

$ (-13,39) $ es solución particular.

\textbf{Solución del homogeneo asociado}

$ 14a+5b = 0 \iff 14a = -5b \implies 14|-5b \implies 14|b \implies b=14k $

$ 14a+5b = 0 \iff 14a = -5b \implies 14a = -5(14k) \implies a = -5k $

Luego $ (-5k, 14k) $ es solución del homogeneo asociado, $ \forall k \in \mathbb{Z} $

\textbf{Solución general}

Uniendo las dos soluciones halladas previamente,

$ S = (-5k, 14k) + (-13,39) = (-5k-13, 14k+39); \forall k \in \mathbb{Z} $

Luego $ a = -5-13 \wedge b= 14k+39 $. Usando el dato de la congruencia,
\begin{align*}
    b\equiv 2a(5) &\iff 14k+39 \equiv 2(-5k-13)(5) \\
    &\iff 4k+4 \equiv 4(5) \\
    &\iff 4(k+1) \equiv 4(5) \\
    &\iff k+1 \equiv 1(5) \\
    &\iff k \equiv 0(5) \\
\end{align*}
Luego se que $ k = 5n; n \in \mathbb{Z} $

Por lo tanto, las soluciones serán $ (a,b) = (-25n-13, 70n+39 ); n \in \mathbb{Z} $

\subsection{Ejercicio 6}

\begin{align*}
    7a \equiv 5(18) &\iff (-5).7.a \equiv (-5).5(18) \\  
    &\iff -35a \equiv -25(18) \\
    &\iff a \equiv 11(18)
\end{align*}
Luego el resto de dividir a $a$ por 18 es 11.

\subsection{Ejercicio 7}
Primero busco los $ (x,y) \in \mathbb{Z}^2: 110x+250y=100 $

\textbf{Verifico que existe solución}

$ (110:250) = 10 \wedge 10|100 \implies $ existe solución.

\textbf{Coprimizar}

Dado que el MCD es disinto a 1, debo coprimizar la ecuación para no perder soluciones.

$ 110x+250y =100 \leftrightsquigarrow 11x+25y=10$

\textbf{Busco una solución particular}

$ (-90,40) $ es solución particular.

\textbf{Solución del homogeneo asociado}

$ 11x+25y = 0 \iff 11x = -25y \implies 11|-25y \implies 11|y \implies y=11k $

$ 11x+25y = 0 \iff 11x = -25y \implies 11x=-25(11k) \implies x = -25k $

Luego $ (-25k, 11k) $ es solución del homogeneo asociado, $ \forall k \in \mathbb{Z} $

\textbf{Solución general}

Uniendo las dos soluciones halladas previamente,

$ S = (-25k, 11k) + (-90,40) = (-25k-90, 11k+40); \forall k \in \mathbb{Z} $

Luego $ x = -25k-90 \wedge y= 11k+40 $,
\begin{align*}
    37^2 |(x-y)^{4321} &\iff 37^2 | (-25k-90-11k-40)^{4321} \\
    &\iff 37^2 | (-36k-130)^{4321} \\
\end{align*}

Pero por propiedades de la divisibilidad,
\begin{align*}
    37^2 | (-36k-130)^{4321} &\iff 37|(-36k-130) \\
    &\iff -36k \equiv 130(37) \\
    &\iff k \equiv 4(37) \\
\end{align*}

Por lo tanto $ (x, y) = (-25n-90, 11n+40) $ para todo $ n \equiv 4(37) $

\subsection{Ejercicio 8}

Sea $ d = (2a-3:4a^2+10a-10) $

Busco llegar a una expresión del tipo $ d|n $ con $ n \in \mathbb{Z} $

Luego,
\begin{align*}
    d|2a-3 \wedge d|4a^2+10a-10 &\iff d|2a(2a-3) - 4a^2-10a+10 \\
    &\iff d|4a^2-6a-4a^2-10a+10 \\
    &\iff d|-16a+10 \\
\end{align*}
Por lo tanto,
\begin{align*}
    \iff d|-16a+10 \wedge d|2a-3 &\iff d|-16a+10+8(2a-3) \\
    &\iff d|-16a+10+10a-24 \\
    &\iff d|-14 \\
    &\iff d \in Div_+(-14) = \{ 1,2,7,14 \} \\
\end{align*}
\begin{itemize}
    \item $ d = 2 \implies 2|2a-3 \implies 2a-3 \equiv 0(2) \implies 0\equiv 3(2) $ ABS
    \item $ d = 7 \implies 7|2a-3 \implies 2a-3 \equiv 0(7) \implies 2a\equiv 3(7) \implies a \equiv 5(7) $
\end{itemize}

Luego con $ a \equiv 5(7) $ se tiene,
\begin{align*}
    4a^2+10a-10 &\equiv 4.25+10.5-10 (7) \\
    &\equiv 2+1+4 (7) \\
    &\equiv 7 (7) \\
    &\equiv 0 (7) \\
\end{align*}

Así, para $ a \equiv 5(7) $ el MCD es igual a 7. No pruebo con 14 ya que $ 14=2.7 $ y si $ 2\not | 2a-3 $ tampoco lo hará 14.

Rta.: Con $ a \equiv 5(7) $ el MCD $ \neq 1 $

\subsection{Ejercicio 9}

Sea $ d = (5a+8:7a+3) $

Busco una expresión del tipo $ d|n $ con $ n \in \mathbb{Z} $
\begin{align*}
    d|5a+8 \wedge d|7a+3 &\iff d|7(5a+8) - 5(7a+3) \\
    &\iff d|35a+56-35a-15 \\
    &\iff d|41 \\
\end{align*}

Luego $ d \in Div_+(41) \iff d \in \{ 1,41 \} $

Con $ d = 41 $,
\begin{align*}
    d = 41 &\implies 41|5a+8 \\
    &\iff 5a+8 \equiv 0 (41) \\
    &\iff 5a\equiv 33 (41) \\
    &\iff 8.5a\equiv 8.33 (41) \\
    &\iff -a\equiv 18 (41) \\
    &\iff a\equiv 23 (41) \\
\end{align*}
Con $ a\equiv 23(41) $ \\
$ 7a+3 \equiv 7.23+3 \equiv 0(41) $

Rta.: $ \begin{cases}
    (5a+8:7a+3) = 41 & a\equiv 23(41) \\
    (5a+8:7a+3) = 1 & a \not \equiv 23(41) \\
\end{cases} $

\subsection{Ejercicio 10}

\subsubsection{Pregunta i}

$ \begin{cases}
    a\equiv 3(10) \\
    a\equiv 2(7) \\
    a\equiv 5(9) \\
\end{cases} $

El TCR me asegura que existe una solución $ x\equiv n (630) $ con $ 0\leq n \leq 630 $ pues 10,7,9 son primos dos a dos.

Quiebro el sistema de ecuaciones en tres.

S1: $ \begin{cases}
    a\equiv 3(10) \\
    a\equiv 0(7) \\
    a\equiv 0(9) \\
\end{cases} $
S2: $ \begin{cases}
    a\equiv 0(10) \\
    a\equiv 2(7) \\
    a\equiv 0(9) \\
\end{cases} $
S3: $ \begin{cases}
    a\equiv 0(10) \\
    a\equiv 0(7) \\
    a\equiv 5(9) \\
\end{cases} $

Busco soluciones a cada sistema por separado.

S1, $ \begin{cases}
    0\equiv 3(10) \\
    0\equiv 0(63) \\
\end{cases} \implies a = 63k \implies 63k \equiv 3(10) \implies 3k \equiv 3(10) \implies k \equiv 1(10)$ 

Luego $ a = 63.k = 63.1 = 63 $

S2, $ \begin{cases}
    a\equiv 2(7) \\
    a\equiv 0(90) \\
\end{cases} \implies a = 90k \implies 90k \equiv 2(7) \implies 6k \equiv 2(7) \implies k \equiv 5(7) $

Luego $ a = 90k = 90.5 = 450 $

S3, $ \begin{cases}
    a\equiv 5(9) \\
    a\equiv 0(70) \\
\end{cases} \implies a = 70k \implies 70k \equiv 5(9) \implies 7k \equiv 5(9) \implies k \equiv 2(9) $

Luego $ a = 70.k = 70.2 = 140 $

Por lo tanto se que $ x\equiv 63+450+140 = 653 (630)$ es solución al sistema.

Rta.: $ x \equiv 653 \equiv 23(630) $ es solución al sistema.

\subsubsection{Pregunta ii}

$ \begin{cases}
    a\equiv 1(6) \\
    a\equiv 2(20) \\
    a\equiv 3(9) \\
\end{cases} 
\iff \begin{cases}
    a\equiv 1(3) \\
    a\equiv 1(2) \\
    a\equiv 2(4) \implies a\equiv 0(2) \\
    a\equiv 2(5) \\
    a\equiv 3(9) \implies a\equiv 0(3) \\
\end{cases} $
Luego el sistema es imcompatible.

\subsubsection{Pregunta iii}
$ \begin{cases}
    a\equiv 1(12) \\
    a\equiv 7(10) \\
    a\equiv 4(9) \\
\end{cases} 
\iff \begin{cases}
    a\equiv 1(3) \\
    a\equiv 1(4) \implies a \equiv 1(2) \\
    a\equiv 7(5) \\
    a\equiv 7(2) \implies a \equiv 1(2) \\
    a\equiv 4(9) \implies a \equiv 1(3) \\
\end{cases}
\iff \begin{cases}
    a\equiv 1(4) \\
    a\equiv 2(5) \\
    a\equiv 4(9) \\
\end{cases} $

Quiebro el sistema en tres.

S1: $ \begin{cases}
    a\equiv 1(4) \\
    a\equiv 0(5) \\
    a\equiv 0(9) \\
\end{cases} $
S2: $ \begin{cases}
    a\equiv 0(4) \\
    a\equiv 2(5) \\
    a\equiv 0(9) \\
\end{cases} $
S3: $ \begin{cases}
    a\equiv 0(4) \\
    a\equiv 0(5) \\
    a\equiv 4(9) \\
\end{cases} $

Busco soluciones a cada sistema,

S1, $ \begin{cases}
    a\equiv 1(4) \\
    a\equiv 0(45) \\
\end{cases} \implies a = 45k \implies 45k \equiv 1(4) \implies k \equiv 1(4) $

Luego $ x_1 = 45 $

S2, $ \begin{cases}
    a\equiv 2(5) \\
    a\equiv 0(36) \\
\end{cases} \implies a = 36k \implies 36k \equiv 2(5) \implies k \equiv 2(5) $

Luego $ x_2 = 36.2 = 72 $

S3, $ \begin{cases}
    a\equiv 4(9) \\
    a\equiv 0(20) \\
\end{cases} \implies a = 20k \implies 20k \equiv 4(9) \implies k \equiv 2(9) $

Luego $ x_3 = 20.2 = 40 $

Entonces sea $ x = x_1 + x_2 + x_3 = 45+72+40 = 157 $

El TCR me asegura que hay una únnica solución del sistema MOD 180

Rta.: $ x \equiv 157 (180) $ es solución al sistema.

\subsection{Ejercicio 11}

\subsubsection{Pregunta i}

$ \begin{cases}
    3a \equiv 4(5) \\
    5a \equiv 4(6) \\
    6a \equiv 2(7) \\
\end{cases} 
\iff \begin{cases}
    3.3a \equiv 3.4(5) \\
    5.5a \equiv 5.4(6) \\
    6.6a \equiv 6.2(7) \\
\end{cases}
\iff \begin{cases}
    a \equiv 3(5) \\
    a \equiv 2(6) \\
    a \equiv 5(7) \\
\end{cases} $

Quiebro el sistema en tres.

S1: $ \begin{cases}
    a \equiv 3(5) \\
    a \equiv 0(6) \\
    a \equiv 0(7) \\
\end{cases} $
S2: $ \begin{cases}
    a \equiv 0(5) \\
    a \equiv 2(6) \\
    a \equiv 0(7) \\
\end{cases} $
S3: $ \begin{cases}
    a \equiv 0(5) \\
    a \equiv 0(6) \\
    a \equiv 5(7) \\
\end{cases} $

Busco soluciones de cada sistema.

S1: $ \begin{cases}
    a \equiv 3(5) \\
    a \equiv 0(42) \\
\end{cases} \implies a = 42k \implies 42k \equiv 3(5) \implies 2k \equiv 3(5) \implies k \equiv 4(5) $

Luego $ x_1 = 42.4 = 168 $

S2: $ \begin{cases}
    a \equiv 2(6) \\
    a \equiv 0(35) \\
\end{cases} \implies a = 35k \implies 35k \equiv 2(6) \implies -k \equiv 2(6) \implies k \equiv 4(6) $

Luego $ x_2 = 35.4 = 140 $

S3: $ \begin{cases}
    a \equiv 5(7) \\
    a \equiv 0(30) \\
\end{cases} \implies a = 30k \implies 30k \equiv 5(7) \implies 2k \equiv 5(7) \implies k \equiv 6(7) $

Luego $ x_3 = 30.6 = 180 $

Así, defino $ x = x_1 + x_2 + x_3 = 168 + 140 +180 = 488 $

Rta.: $ a\equiv 488 \equiv 68 (210) $ es solución al sistema.

\subsubsection{Pregunta ii}

$ \begin{cases}
    3a \equiv 1(10) \\
    5a \equiv 3(6) \\
    9a \equiv 1(14) \\
\end{cases} 
\iff \begin{cases}
    -a \equiv 3(10) \implies a \equiv 7(10) \\
    a \equiv 3(6) \\
    -a \equiv 3(14) \implies a \equiv 11(14) \\
\end{cases}
\iff \begin{cases}
    a \equiv 7(10) \\
    a \equiv 3(6) \\
    a \equiv 11(14) \\
\end{cases}
\iff \begin{cases}
    a \equiv 7(5) \\
    a \equiv 7(2) \\
    a \equiv 3(3) \\
    a \equiv 3(2) \\
    a \equiv 11(7) \\
    a \equiv 11(2) \\
\end{cases}
\iff \begin{cases}
    a \equiv 7(5) \\
    a \equiv 0(3) \\
    a \equiv 11(14) \\
\end{cases} $

Quiebro el sostema en tres.

S1: $ \begin{cases}
    a \equiv 7(5) \\
    a \equiv 0(3) \\
    a \equiv 0(14) \\
\end{cases} $
S2: $ \begin{cases}
    a \equiv 0(5) \\
    a \equiv 0(3) \\
    a \equiv 0(14) \\
\end{cases} $
S3: $ \begin{cases}
    a \equiv 0(5) \\
    a \equiv 0(3) \\
    a \equiv 11(14) \\
\end{cases} $

Busco soluciones de cada sistema.

S1: $ \begin{cases}
    a \equiv 7(5) \\
    a \equiv 0(42) \\
\end{cases} \implies a = 42k \implies 42k \equiv 7(5) \implies 2k \equiv 2(5) \implies k \equiv 1(5) $

Luego $ x_1 = 42k = 42.1 = 42 $

S2: $ \begin{cases}
    a \equiv 0(210) \\
\end{cases} $

Luego $ x_2 = 0 $

S3: $ \begin{cases}
    a \equiv 11(14) \\
    a \equiv 0(15) \\
\end{cases} \implies a = 15k \implies 15k \equiv 11(14) \implies k \equiv 11(14) $

Luego $ x_3 = 15k = 15.11 = 165 $

Por lo tanto, $ x = x_1 + x_2 + x_3 = 42 + 0 + 165 = 207 $

Rta.: $ a\equiv 207(210) $

\subsubsection{Pregunta iii}

$ \begin{cases}
    15a \equiv 10(35) \\
    21a \equiv 15(8) \\
    18a \equiv 24(30) \\
\end{cases} 
\iff \begin{cases}
    3a \equiv 2(7) \\
    21a \equiv 15(8) \\
    3a \equiv 4(5) \\
\end{cases}
\iff \begin{cases}
    -a \equiv 4(7) \implies a \equiv 3(7) \\
    a \equiv 3(8) \\
    -a \equiv 12(5) \implies a \equiv 3(5) \\
\end{cases}
\iff \begin{cases}
    a\equiv 3(7) \\
    a\equiv 3(8) \\
    a\equiv 3(5) \\
\end{cases} $

Rta.: $ a\equiv 3(280) $

\subsection{Ejercicio 12}

\subsubsection{Pregunta i}

$ \begin{cases}
    a\equiv 5(6) \\
    a\equiv 3(10) \\
    a\equiv 5(8) \\
\end{cases} 
\iff \begin{cases}
    a\equiv 5(3) \implies a\equiv 2(3) \\
    a\equiv 5(2) \implies a\equiv 1(2) \\
    a\equiv 3(5) \\
    a\equiv 3(2) \implies a\equiv 1(2) \\
    a\equiv 5(8) \\
\end{cases}
\iff \begin{cases}
    a\equiv 2(3) \\
    a\equiv 3(5) \\
    a\equiv 5(8) \\
\end{cases} $

Divido el sistema en tres.

S1: $ \begin{cases}
    a\equiv 2(3) \\
    a\equiv 0(5) \\
    a\equiv 0(8) \\
\end{cases} $
S2: $ \begin{cases}
    a\equiv 0(3) \\
    a\equiv 3(5) \\
    a\equiv 0(8) \\
\end{cases} $
S3: $ \begin{cases}
    a\equiv 0(3) \\
    a\equiv 0(5) \\
    a\equiv 5(8) \\
\end{cases} $

Busco soluciones a cada sistema.

S1: $ \begin{cases}
    a\equiv 2(3) \\
    a\equiv 0(40) \\
\end{cases} \implies a = 40k \implies 40k \equiv 2(3) \implies k \equiv 2(3) $

Luego $ x_1 = 40.2 = 80 $

S2: $ \begin{cases}
    a\equiv 3(5) \\
    a\equiv 0(24) \\
\end{cases} \implies a = 24k \implies 24k \equiv 3(5) \implies -k \equiv 3(5) \implies k \equiv 2(5) $

Luego $ x_2 = 24.2 = 48 $

S3: $ \begin{cases}
    a\equiv 5(8) \\
    a\equiv 0(15) \\
\end{cases} \implies a = 15k \implies 15k \equiv 5(8) \implies k \equiv 3(8) $

Luego $ x_3 = 15.3 = 45 $

Por lo tanto, $ x = x_1 + x_2 + x_3 = 80+48+45 = 173 $

Rta.: $ r_{480}(a) = 173 $

\subsubsection{Pregunta ii}

$ \begin{cases}
    a\equiv 13(21) \\
    6a\equiv 9(15) \\
\end{cases} 
\iff \begin{cases}
    a \equiv 13(21) \\
    2a \equiv 3(5) \\
\end{cases}
\iff \begin{cases}
    a \equiv 13(21) \\
    a \equiv 4(5) \\
\end{cases} $

Divido el sistema en dos.

S1: $ \begin{cases}
    a \equiv 13(21) \\
    a \equiv 0(5) \\
\end{cases} $
S2: $ \begin{cases}
    a \equiv 0(21) \\
    a \equiv 5(5) \\
\end{cases} $

Busco soluciones a los sistemas.

S1: $ \begin{cases}
    a \equiv 13(21) \\
    a \equiv 0(5) \\
\end{cases} \implies a = 5k \implies 5k \equiv 13(21) \implies k \equiv 11(21) $

Luego $ x_1 = 5k = 5.11 = 55 $   

S2: $ \begin{cases}
    a \equiv 4(5) \\
    a \equiv 0(21) \\
\end{cases} \implies a = 21k \implies 21k \equiv 4(5) \implies k \equiv 4(5) $

Luego $ x_2 = 21k = 21.4 = 84 $   

Por lo tanto $ x = x_1 + x_2 = 55+84 = 139 \implies a \equiv 35(105)$

Rta.: 34 es el entero positivo más chico que cumple lo pedido.

\subsection{Ejercicio 13}

$ \begin{cases}
    a \equiv 4(12) \\
    a \equiv 43(63) \\
\end{cases} 
\iff \begin{cases}
    a \equiv 4(3) \\
    a \equiv 4(4) \\
    a \equiv 43(9) \\
    a \equiv 43(7) \\
\end{cases}
\iff \begin{cases}
    a \equiv 1(3) \\
    a \equiv 0(4) \\
    a \equiv 7(9) \implies a \equiv 1(3) \\
    a \equiv 1(7) \\
\end{cases}
\iff \begin{cases}
    a\equiv 0(4) \\
    a\equiv 7(9) \\
    a\equiv 1(7) \\
\end{cases} $

Divido el sistema en tres.

S1: $ \begin{cases}
    a\equiv 0(4) \\
    a\equiv 0(9) \\
    a\equiv 0(7) \\
\end{cases} $
S2: $ \begin{cases}
    a\equiv 0(4) \\
    a\equiv 7(9) \\
    a\equiv 0(7) \\
\end{cases} $
S3: $ \begin{cases}
    a\equiv 0(4) \\
    a\equiv 0(9) \\
    a\equiv 1(7) \\
\end{cases} $

Busco soluciones a los sistemas.

S1: $ \begin{cases}
    a\equiv 0(252) \\
\end{cases} $

Luego $ x_1 = 0 $

S2: $ \begin{cases}
    a\equiv 7(9) \\
    a\equiv 0(28) \\
\end{cases} \implies a = 28k \implies 28k \equiv 7(9) \implies k \equiv 7(9) $

Luego $ x_2 = 28k = 28.7 = 196$

S3: $ \begin{cases}
    a\equiv 1(7) \\
    a\equiv 0(36) \\
\end{cases} \implies a = 36k \implies 36k \equiv 1(7) \implies k \equiv 1(7) $

Luego $ x_3 = 36k = 36.1 = 36$

Por lo tanto, $ x = x_1 + x_2 + x_3 = 0+196+36 = 232 \implies a \equiv 232(252) $

Luego $ a \equiv 232(252) \iff a = 252k+232 $

Ahora uso que $ 12600 \leq a \leq 13300 $,
\begin{align*}
    12600 &\leq a \leq 13300 \\
    12600 &\leq 252k+232 \leq 13300 \\
    \frac{12600-232}{252} &\leq k \leq \frac{13300-232}{252} \\
    49,07 &\leq k \leq 51,85 \\
\end{align*}

Luego $ k \in \{ 50,51 \} $

\begin{itemize}
    \item $ k = 50 \implies a = 12832 $
    \item $ k = 51 \implies a = 13084 $
\end{itemize}

Rta.: Había 12832 o 13084 latas.

\subsection{Ejercicio 14}

$ a^2 \equiv 21(238) \iff \begin{cases}
    a^2 \equiv 21(2) \\
    a^2 \equiv 21(7) \\
    a^2 \equiv 21(17) \\
\end{cases} 
\iff \begin{cases}
    a^2 \equiv 1(2) \\
    a^2 \equiv 0(7) \\
    a^2 \equiv 4(17) \\
\end{cases} $

Usando tabla de restos llego a

$ \begin{cases}
    a \equiv 1(2) \\
    a \equiv 0(7) \\
    a \equiv 2(17) \vee a\equiv 15(17) \\
\end{cases} $

Divido el sistema en tres.

S1: $ \begin{cases}
    a \equiv 1(2) \\
    a \equiv 0(7) \\
    a \equiv 0(17) \vee a\equiv 0(17) \\
\end{cases} $
S2: $ \begin{cases}
    a \equiv 0(2) \\
    a \equiv 0(7) \\
    a \equiv 0(17) \vee a\equiv 0(17) \\
\end{cases} $
S3: $ \begin{cases}
    a \equiv 0(2) \\
    a \equiv 0(7) \\
    a \equiv 2(17) \vee a\equiv 15(17) \\
\end{cases} $

Busco soluciones de cada sistema.

S1: $ \begin{cases}
    a \equiv 1(2) \\
    a \equiv 0(119) \\
\end{cases} \implies a = 119k \implies 119k \equiv 1(2) \implies k \equiv 1(2) $

Luego $ x_1 = 119 $

S2: $ \begin{cases}
    a \equiv 0(238) \\
\end{cases} $

Luego $ x_2 = 0 $

S3a: $ \begin{cases}
    a \equiv 2(17) \\
    a \equiv 0(14) \\
\end{cases} \implies a = 14k \implies 14k \equiv 2(17) \implies k \equiv 5(17) $

Luego $ x_{3a} = 14.5 = 70 $

S3b: $ \begin{cases}
    a \equiv 15(17) \\
    a \equiv 0(14) \\
\end{cases} \implies a = 14k \implies 14k \equiv 15(17) \implies k \equiv 12(17) $

Luego $ x_{3b} = 14.12 = 168 $

Así, obtengo dos soluciones:
\begin{itemize}
    \item $ x_a = x_1 + x_2 + x_{3a} = 119+0+70 = 189 $
    \item $ x_b = x_1 + x_2 + x_{3b} = 119+0+168 = 287 \equiv 49(238) $
\end{itemize}

Rta.: Los posibles restos son 49 y 189.

\end{document}
