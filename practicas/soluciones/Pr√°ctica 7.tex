\documentclass{article}
\usepackage{ifthen}
\usepackage{amssymb}
\usepackage{multicol}
\usepackage{graphicx}
\usepackage[absolute]{textpos}
\usepackage{amsmath, amscd, amssymb, amsthm, latexsym}
\usepackage{xspace,rotating,dsfont,ifthen}
\usepackage[spanish,activeacute]{babel}
\usepackage[utf8]{inputenc}
\usepackage{pgfpages}
\usepackage{pgf,pgfarrows,pgfnodes,pgfautomata,pgfheaps,xspace,dsfont}
\usepackage{listings}
\usepackage{multicol}
\usepackage{todonotes}
\usepackage{url}
\usepackage{float}
\usepackage{framed,mdframed}
\usepackage{cancel}

\usepackage[strict]{changepage}


\makeatletter


\newcommand\hfrac[2]{\genfrac{}{}{0pt}{}{#1}{#2}} %\hfrac{}{} es un \frac sin la linea del medio

\newcommand\Wider[2][3em]{% \Wider[3em]{} reduce los m\'argenes
\makebox[\linewidth][c]{%
  \begin{minipage}{\dimexpr\textwidth+#1\relax}
  \raggedright#2
  \end{minipage}%
  }%
}


\@ifclassloaded{beamer}{%
  \newcommand{\tocarEspacios}{%
    \addtolength{\leftskip}{4em}%
    \addtolength{\parindent}{-3em}%
  }%
}
{%
  \usepackage[top=1cm,bottom=2cm,left=1cm,right=1cm]{geometry}%
  \usepackage{color}%
  \newcommand{\tocarEspacios}{%
    \addtolength{\leftskip}{3em}%
    \setlength{\parindent}{0em}%
  }%
}

\usepackage{caratula}
\usepackage{enumerate}
\usepackage{hyperref}
\usepackage{graphicx}
\usepackage{amsfonts}
\usepackage{enumitem}
\usepackage{amsmath}

\decimalpoint
\hypersetup{colorlinks=true, linkcolor=black, urlcolor=blue}
\setlength{\parindent}{0em}
\setlength{\parskip}{0.5em}
\setcounter{tocdepth}{2} % profundidad de indice
\setcounter{section}{6} % nro de section
\renewcommand{\thesubsubsection}{\thesubsection.\Alph{subsubsection}}
\graphicspath{ {images/} }

% End latex config

\begin{document}

\titulo{Práctica 7}
\fecha{2do cuatrimestre 2021}
\materia{Álgebra I}
\integrante{Yago Pajariño}{546/21}{ypajarino@dc.uba.ar}

%Carátula
\maketitle
\newpage

%Indice
\tableofcontents
\newpage

% Aca empieza lo propio del documento
\section{Práctica 7}

\subsection{Ejercicio 1}

Rdo. propiedades del producto y suma de polinomios: 
\begin{itemize}
    \item Grado de un producto de polinomios $ gr(ab) = gr(a) + gr(b) $
    \item Coeficiente principal de un prodcuto de polinomios $ cp(ab) = ca(a)\cdot cd(b) $
    \item $ \begin{cases}
        gr(f+g) \leq max(gr(f); gr(g)) \\
        gr(f+g) = max(gr(f); gr(g)) \iff gr(f) \neq gr(g) \vee (gr(f) = gr(g) \wedge cp(f) \neq cp(g))\\
    \end{cases}  $
\end{itemize}

\subsubsection{Pregunta i}

\begin{itemize}
    \item $ gr(p) = 77 . gr(4x^6-2x^5+3x^2-2x+7) = 77.6 = 462 $
    \item $ cp(p) = 4^{77} $ 
\end{itemize}

\subsubsection{Pregunta ii}

Sea $ p = a^4 - b^7 $ con $ a = -3x^7 + 5x^3 + x^2 - x + 5 $ y $ b = 6x^4 + 2x^3 + x - 2 $
\begin{align*}
    gp(p) &= max(gr(a^4); gr(b^7)) \iff gr(a^4) \neq gr(b^7) \vee cp(a^4) \neq cp(b^7) \\
    &= max(7.4; 4.7) \iff cp(a^4) \neq cp(b^7) \\
    &= 28 \iff (-3)^4 \neq 6^7 \\
    &= 28 \iff 81 \neq 279936
\end{align*}
\begin{itemize}
    \item $ gr(p) = 28 $
    \item $ cp(p) = 81-6^7 $ 
\end{itemize}

\subsubsection{Pregunta iii}

Sea $ p = a - b + c $ con $ \begin{cases}
    a = (-3x^5 + x^4 - x + 5)^4 \\
    b = 82x^{20} \\
    c = 19x^{19}
\end{cases} $

Luego $ p = 81x^{20} + (...) - 81x^{20}+19x^{19} \implies gr(p) = 19 $ pues se cancelan los termino con $ x^{20} $

Entonces busco el coeficiente para $ x^{19} $
\begin{align*}
    cp(p) &= a_{19} + b_{19} + c_{19} \\
    &= (-3.-3.-3.1) + 0 + 19 \\
    &= -27 + 0 + 19 \\
    &= -8 \\
\end{align*}
\begin{itemize}
    \item $ gr(p) = 19 $
    \item $ cp(p) = -8 $ 
\end{itemize}

\subsection{Ejercicio 2}

\begin{enumerate}
    \item \begin{enumerate}
        \item En $ \mathbb{Q}[x] = 2 $
        \item En $ \frac{\mathbb{Z}}{2\mathbb{Z}}[x] = 2 $
    \end{enumerate}
    \item Usando bin de Newton, $ c(20) = \binom{133}{20}(3i)^{113} $
    \item Usando bin de Newton cuatro veces, \begin{itemize}
        \item $ a_1 = \binom{4}{1}x^{1}(-1)^3 \cdot \binom{19}{19}x^{19}5^0 = -4x^{20} $
        \item $ a_2 = \binom{4}{2}x^{2}(-1)^2 \cdot \binom{19}{18}x^{18}5^1 = 570x^{20} $
        \item $ a_3 = \binom{4}{3}x^{3}(-1)^1 \cdot \binom{19}{17}x^{17}5^2 = -17100x^{20} $
        \item $ a_4 = \binom{4}{4}x^{4}(-1)^0 \cdot \binom{19}{16}x^{16}5^3 = 121125x^{20} $
    \end{itemize}
    Luego $ c(20) = a_1 + a_2 + a_3 + a_4 - 5 = -4+570-17100+121125 -5 = 104586 $
    \item $ c(20) = 21504 $
\end{enumerate}

\subsection{Ejercicio 3}

\subsubsection{Pregunta i}

Reescribo el polinomio que me dan,
\begin{align*}
    f^2 = xf + x + 1 &\iff f^2 - xf = x+1 \\
    &\iff f(f-x) = x+1 \\
    &\iff f\neq 0 \wedge f-x \neq 0 \\
\end{align*}
Tomo grado a ambos lados,
\begin{align*}
    gr(f) + gr(f-x) &= gr(x+1) \\
    gr(f) + gr(f-x) &= 1 \\
\end{align*}
Luego el grado de f tiene que se menor a 2.

\textbf{Caso gr(f) = 1}

Si $ gr(f) = 1 \implies gr(f-x) = 0 $ para cumplir la igualdad de grados.

Luego f es de la forma $ f = ax+b $ con $ a = 1 $

Entonces,
\begin{align*}
    f(f-x) = x+1 &\iff (x+b)(x+b-x) = x+1 \\
    &\iff xb + b^2 = x+1 \\
    &\iff \text{Por igualdad de polinomios} \begin{cases}
        b = 1 \\
        b^2 = 1 \\
    \end{cases} \iff b = 1
\end{align*}
Así, $ f_1 = x+1 $

\textbf{Caso gr(f) = 0}

Que el grado del polinomio sea igual a cero implica que $ f = c $ con c una constante.

Entonces,
\begin{align*}
    f(f-x) = x+1 &\iff c(c-x) = x+1 \\
    &\iff c^2 - cx = x+1 \\
    &\iff \text{Por igualdad de polinomios} \begin{cases}
        -c = 1 \\
        c^2 = 1
    \end{cases} \implies c = -1
\end{align*}
Así, $ f_2 = -1 $

Rta.: $ f = x+1 $ y $ f=-1 $

\subsubsection{Pregunta ii}

Reescribo el polinomio que me dan,
\begin{align*}
    f^2 - xf = x^2 + 1 \iff f(f-x) = -x^2 + 1
\end{align*}
Tomo grado a ambos lados de la igualdad.
\begin{align*}
    gr(f) + gr(f-x) &= gr(-x^2 + 1) \\
    0 + 2 &= 2 \text{ No puede ser} \\
    1 + 1 &= 2 \\
    2 + 0 &= 2 \text{ No puede ser} \\
\end{align*}
Así, el único caso posible es que $ gr(f) = 1 $ y que $ gr(f-x) = 1 $

Sea $ f = ax+b $,
\begin{align*}
    f(f-x) = -x^2 + 1 &\iff (ax+b)(ax+b-x) = -x^2+1 \\
    &\iff (ax+b)((a-1)x+b) = -x^2+1 \\
    &\iff a(a-1)x^2 + abx + b(a-1)x + b^2 = -x^2+1 \\
    &\iff a(a-1)x^2 + (ab+b(a-1))x + b^2 = -x^2+1 \\
    &\iff \text{Por igualdad de polinomios }\begin{cases}
        a(a-1) = -1 \\
        ab+b(a-1) = 0 \\
        b^2 = -1 
    \end{cases}
\end{align*}
Busco soluciones para el sistema de tres ecuaciones que resultó.

De la tercera, se que $ b = \pm 1 $

$ b = 1 \implies a + a + 1 = 0 \iff 2a = 1 \iff a = \frac{1}{2} $

Pero con $ a = \frac{1}{2} \wedge b= 1 \implies \frac{1}{2} (\frac{1}{2}-1) = \frac{1}{4} - \frac{1}{2} = -\frac{1}{4} \neq -1 $

Luego $ b = 1 $ NO sirve.

$ b = -1 \implies -a-a+1 = 0 \implies -2a = -1 \implies a = \frac{1}{2} $

Se llega al mismo valor de a que con b=1 y ya se probó que no sirve.

Por lo tanto, $ \not \exists f \in \mathbb{C}[x] $ que cumpla lo pedido.

\subsubsection{Pregunta iii}

Reescribo el polinomio que me dan,
\begin{align*}
    (x+1)f^2 = x^6 + xf &\iff (x+1)f^2 - xf = x^6 \\
    &\iff f((x+1)f - x) = x^6 \\
\end{align*}
Aplico grado a ambos lados de la igualdad.
\begin{align*}
    gr(f) + gr((x+1)f - x) &= gr(x^6) \\
    0 + 6 &= 6 \\
    1 + 5 &= 6 \\
    2 + 4 &= 6 \\
    3 + 3 &= 6 \\
    4 + 2 &= 6 \\
    5 + 1 &= 6 \\
    6 + 0 &= 6 \\
\end{align*}
Luego de dar todos los posibles valores a $ gr(f) $, se puede ver que no existe $ gr((x+1)f - x) $ que cumpla lo pedido.

Por lo tanto, $ \not \exists f \in \mathbb{C}[x] $ que cumpla lo pedido.

\subsubsection{Pregunta iv}

Dado que por enunciado se que $ f \neq 0 $, puedo reescribir la igualdad como,
\begin{align*}
    f^3 = gr(f) \cdot x^2f &\iff f^2 = gr(f)\cdot x^2 \\
\end{align*}
Aplico grado a ambos lados de la igualdad.
\begin{align*}
    gr(f^2) &= gr(gr(f)\cdot x^2) \\
    gr(f^2) &= 2 \\
    gr(f.f) &= 2 \\
    2gr(f) &= 2 \\
    gr(f) &= 1 \\
\end{align*}
Luego, con $ f = ax+b $,
\begin{align*}
    f^2 = gr(f)\cdot x^2 &\iff (ax+b)^2 = x^2 \\
    &\iff a^2x^2 + 2abx + b^2 = x^2 \\
    &\iff \text{Por igualdad de polinomios} \begin{cases}
        a^2 = 1 \\
        2ab = 0 \\
        b = 0
    \end{cases}
\end{align*}
Entonces, $ a = \pm 1 $ y $ b = 0 $ son las soluciones del sistema.

Rta.: $ f_1 = x $ y  $ f_2 = -x $ son los únicos polinomios que cumplen lo pedido.



\end{document}
