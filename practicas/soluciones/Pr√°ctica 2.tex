\documentclass{article}
\usepackage{ifthen}
\usepackage{amssymb}
\usepackage{multicol}
\usepackage{graphicx}
\usepackage[absolute]{textpos}
\usepackage{amsmath, amscd, amssymb, amsthm, latexsym}
\usepackage{xspace,rotating,dsfont,ifthen}
\usepackage[spanish,activeacute]{babel}
\usepackage[utf8]{inputenc}
\usepackage{pgfpages}
\usepackage{pgf,pgfarrows,pgfnodes,pgfautomata,pgfheaps,xspace,dsfont}
\usepackage{listings}
\usepackage{multicol}
\usepackage{todonotes}
\usepackage{url}
\usepackage{float}
\usepackage{framed,mdframed}
\usepackage{cancel}

\usepackage[strict]{changepage}


\makeatletter


\newcommand\hfrac[2]{\genfrac{}{}{0pt}{}{#1}{#2}} %\hfrac{}{} es un \frac sin la linea del medio

\newcommand\Wider[2][3em]{% \Wider[3em]{} reduce los m\'argenes
\makebox[\linewidth][c]{%
  \begin{minipage}{\dimexpr\textwidth+#1\relax}
  \raggedright#2
  \end{minipage}%
  }%
}


\@ifclassloaded{beamer}{%
  \newcommand{\tocarEspacios}{%
    \addtolength{\leftskip}{4em}%
    \addtolength{\parindent}{-3em}%
  }%
}
{%
  \usepackage[top=1cm,bottom=2cm,left=1cm,right=1cm]{geometry}%
  \usepackage{color}%
  \newcommand{\tocarEspacios}{%
    \addtolength{\leftskip}{3em}%
    \setlength{\parindent}{0em}%
  }%
}

\usepackage{caratula}
\usepackage{enumerate}
\usepackage{hyperref}
\usepackage{graphicx}
\usepackage{amsfonts}
\usepackage{enumitem}
\usepackage{amsmath}

\decimalpoint
\hypersetup{colorlinks=true, linkcolor=black, urlcolor=blue}
\setlength{\parindent}{0em}
\setlength{\parskip}{0.5em}
\setcounter{tocdepth}{2} % profundidad de indice
\setcounter{section}{0} % nro de section
\renewcommand{\thesubsubsection}{\thesubsection.\Alph{subsubsection}}
\graphicspath{ {images/} }

% End latex config

\begin{document}

\titulo{Práctica 2}
\fecha{2do cuatrimestre 2021}
\materia{Álgebra I}
\integrante{Yago Pajariño}{546/21}{ypajarino@dc.uba.ar}

%Carátula
\maketitle
\newpage

%Indice
\tableofcontents
\newpage

% Aca empieza lo propio del documento
\section{Práctica 2}

\subsection{Ejercicio 1}

\begin{enumerate}
    \item \begin{enumerate}[label=(\alph*)]
        \item $\sum_{i=1}^{100}i$
        \item $\sum_{i=1}^{10}i^2$
        \item $\sum_{i=1}^{12}(-1)^i . i^2$
        \item $\sum_{i=1 \wedge \text{i impar}}^{21}i^2$
        \item $\sum_{i=0}^{n}2i+1$
        \item $\sum_{i=1}^{n}i.n$
    \end{enumerate}
    \item \begin{enumerate}[label=(\alph*)]
        \item $\frac{100!}{4!}$
        \item $\prod_{i=1}^{10}2^i$
        \item $\prod_{i=1}^{n}i.n$
    \end{enumerate}
\end{enumerate}

\subsection{Ejercicio 2}
\begin{enumerate}[label=(\alph*)]
    \item $2+4$ ; $2(n-6)+ 2(n-5)$
    \item $\frac{1}{n(n+1)} + \frac{1}{(n+1)(n+2)}$ ; $\frac{1}{(2n-1).2n} + \frac{1}{2n.(2n+1)}$
    \item $\frac{n+1}{2} + \frac{n+2}{4}$ ; $\frac{n+n-1}{2.(n-1)} + \frac{2n}{2n}$
    \item $n + \frac{n}{2}$ ; $\frac{n}{n^2-1} + \frac{n}{n^2}$
    \item $-(n+1) . (n+2)$ ; $\frac{n+n-1}{2(n-1)-3} . \frac{2n}{2n-3}$
\end{enumerate}

\subsection{Ejercicio 3}
\begin{enumerate}[label=(\alph*)]
    \item \begin{align*}
        \sum_{i=1}^{n}(4i+1) &= \sum_{i=1}^{n}4i + \sum_{i=1}^{n} 1 \\
        &= 4 . \sum_{i=1}^{n}i + n \\
        &= 4 . \frac{n.(n+1)}{2} + n \\
        &= 2 . n(n+1) + n \\
        &= 2n^2 + 3n
    \end{align*}
    \item \begin{align*}
        \sum_{i=6}^{n}2(i-5) &= 2\sum_{i=6}^{n}(i-5)\\
        &= 2\left( \sum_{i=6}^{n}i -\sum_{i=6}^{n}5 \right) \\
        &= 2\left( \left(\sum_{i=1}^{n}i - \sum_{i=1}^{5}i\right) -\left(\sum_{i=1}^{n}5 - \sum_{i=1}^{5}5\right) \right) \\
        &= 2\left( \left(\frac{n(n+1)}{2} - 15 \right) -\left(5n - 25 \right)\right) \\
        &= 2\left( \left(\frac{n(n+1) - 30}{2} \right) - 5(n+5)\right) \\
        &= n(n+1) - 30 - 10(n+5) \\ 
        &= n^2 + n - 30 - 10n-50 \\
        &= n^2 - 9n-80 \\
    \end{align*}
\end{enumerate}

\subsection{Ejercicio 4}
\begin{enumerate}[label=(\alph*)]
    \item $\sum_{i=0}^{n}2^i = n^{n+1}-1$
    \item $\sum_{i=1}^{n}q^i = \begin{cases}
        \frac{q^{n+1}-2}{q-1} & q \neq 1 \\
        n & q = 1 \\
    \end{cases}$
    \item $\sum_{i=1}^{n}q^{2i} = \sum_{i=1}^{n}(q^2)^i =  \begin{cases}
        \frac{(q^2)^{n+1}-1}{q^2-1} & q \neq 1 \\
        n+1 & q = 1 \\
    \end{cases}$
    \item $\sum_{i=n}^{2n}q^{i} = \sum_{i=0}^{2n}q^{i} - \sum_{i=0}^{n-1}q^{i} = \begin{cases}
        \frac{q^{2n+1} - q^n}{q-1} & q \neq 1 \\
        n+1 & q =1
    \end{cases}$
\end{enumerate}

\subsection{Ejercicio 5}
Usando la suma aritmética:
\begin{align*}
    \sum_{i=1}^{n}(2i-1) &= \sum_{i=1}^{n}2i - \sum_{i=1}^{n}1 \\
    &= 2.\frac{n(n+1)}{2} - n \\
    &= n^2 + n - n \\
    &= n^2\\
\end{align*}

Usando el principio de inducción:

Defino el predicado $p(n): \sum_{i=1}^{n}(2i-1) = n^2; \forall n \in \mathbb{N}$

\underline{Caso base n = 1}

$\sum_{i=1}^{1}(2i-1) = 2 - 1 = 1$

$n^2 = 1^2 = 1$

Luego el caso base es verdadero.

\underline{Paso inductivo}

Quiero probar que para $k \geq 1$, $p(k) \implies p(k+1)$

HI: $\sum_{i=1}^{k}(2i-1) = k^2$

QpQ: $\sum_{i=1}^{k+1}(2i-1) = (k+1)^2$

Pero,

\begin{align*}
    \sum_{i=1}^{k+1}(2i-1) &= \sum_{i=1}^{k}(2i-1) + 2(k+1)-1 \\
    &= k^2 + 2(k+1)-1 \\
    &= k^2 + 2k+1 \\
    &= (k+1)^2
\end{align*}

Luego el paso inductivo es verdadero. Por lo tanto $p(n)$ es verdadero, $\forall n \in \mathbb{N}$

\subsection{Ejercicio 6}

\subsubsection{Pregunta i}

\underline{Prueba por inducción:}

Defino el predicado $p(n) : \sum_{i=1}^{n}i^2 = \frac{n(n+1)(2n+1)}{6}$

\textbf{Caso base n = 1}

$\sum_{i=1}^{1}i^2 = 1^2 = 1$

$\frac{n(n+1)(2n+1)}{6} = \frac{1.2.3}{6} = \frac{6}{6} = 1$

Luego el caso base $p(1)$ es verdadero.

\textbf{Paso inductivo}

Para todo $k \geq 1: p(k) \implies (k+1)$

HI: $\sum_{i=1}^{k}i^2 =  \frac{k(k+1)(2k+1)}{6}$

QpQ: $\sum_{i=1}^{k+1}i^2 =  \frac{(k+1)(k+2)(2(k+1)+1)}{6} =  \frac{(k+1)(k+2)(2k+3)}{6}$

Pero,

\begin{align*}
    \sum_{i=1}^{k+1}i^2 &= \sum_{i=1}^{k}i^2 + (k+1)^2 \\
    &= \frac{k(k+1)(2k+1)}{6} + (k+1)^2 \\
\end{align*}

Entonces necesito probar que, 

\begin{align*}
    \frac{k(k+1)(2k+1)}{6} + (k+1)^2 &= \frac{(k+1)(k+2)(2k+3)}{6} \\
    \iff \frac{k(k+1)(2k+1) + 6(k+1)^2}{6} &= \frac{(k+1)(k+2)(2k+3)}{6} \\
    \iff k(2k+1) + 6(k+1) &= (k+2)(2k+3) \\
    \iff 2k^2+k + 6k+6 &= 2k^2+3k+4k+6 \\
    \iff 2k^2+7k+6 &= 2k^2+7k+6 \\
\end{align*}

Luego $\sum_{i=1}^{k+1}i^2 =  \frac{(k+1)(k+2)(2k+3)}{6}$ como se quería probar, el paso inductivo el verdadero.

Por lo tanto, $p(n)$ es verdadero, $\forall n \in \mathbb{N}$.

\subsubsection{Pregunta ii}

Defino el predicado $p(n) : \sum_{i=1}^{n}i^3 = \frac{n^2(n+1)^2}{4}$

\textbf{Caso base n = 1}

$\sum_{i=1}^{n}i^3 = 1^3 = 1$

$\frac{n^2(n+1)^2}{4} = \frac{4}{4} = 1$

Luego el caso base $p(1)$ es verdadero.

\textbf{Paso inductivo}

Para todo $k \geq 1: p(k) \implies (k+1)$

HI: $\sum_{i=1}^{k}i^3 = \frac{k^2(k+1)^2}{4}$

QpQ: $\sum_{i=1}^{k+1}i^3 = \frac{(k+1)^2(k+2)^2}{4}$

Pero,
\begin{align*}
    \sum_{i=1}^{k+1}i^3 &= \sum_{i=1}^{k}i^3 + (k+1)^3 \\
    &= \frac{k^2(k+1)^2}{4} + (k+1)^3 \\
    &= \frac{k^2(k+1)^2 + 4(k+1)^3}{4}
\end{align*}

Luego debo probar, 
\begin{align*}
    \frac{k^2(k+1)^2 + 4(k+1)^3}{4} &= \frac{(k+1)^2(k+2)^2}{4} \\
    \iff k^2 + 4(k+1) &= k^2 + 4k + 4 \\
    \iff k^2 + 4k + 4 &= k^2 + 4k + 4 \\
\end{align*}

Luego $p(k) \implies p(k+1); \forall k \geq 1$, como se quería probar.

Por lo tanto, $p(n)$ es verdadero, $\forall n \in \mathbb{N}$.

\end{document}
